\documentclass{resume}
\usepackage{zh_CN-Adobefonts_external}
\usepackage{linespacing_fix} % disable extra space before next section
\usepackage{cite}

\begin{document}
\pagestyle{empty}

\name{高正炎}
\job{区块链/后端开发工程师}
\contactInfo{{moyishizhe@gmail.com}}{18404968744}{https://dbqf.xyz}
\section{工作经历}
\datedsubsection{\textbf{比特大陆,硅芯扬航} , \href{https://btc.com}{BTC.com}}{2017 年 4 月 -- 2022 年 5 月}
\begin{onehalfspacing}
\role{负责区块链浏览器后端业务开发及架构,主要技术栈:Go,PHP,C++,Python}{2019 - 2022}
\begin{itemize}
  \item 负责开发浏览器 Parser、后端 API 、实时计算、爬虫、数据检查和修复等模块,梳理相关架构
  \item 开发和维护 BTC 系列 Parser,修改相关币种代码,优化未确认交易处理
  \item 开发和重构 Ethereum Parser,为此开发新架构版本的 Parser,支持多币种、性能优先、插件化,解析链上数据、合约数据、NFT 数据到数据库 MySQL 和 ClickHouse 等
  \item 探索和实践浏览器实时计算,使用 PyFlink 和 ClickHouse 解决一些业务需求,使用 Flink SQL 来达到流批一体,并在 \href{https://2020.flink-forward.org.cn/}{Flink Forward Asia 2020} 发表实时计算的相关演讲
  \item 使用微服务框架 go-micro 重新开发了浏览器业务的 API,按业务拆分模块,使用一些自动化生成工具和模板,尽可能地减少程式化编码
\end{itemize}
\role{负责矿池业务后端开发及架构,主要技术栈:Go,PHP,Nodejs}{2017 - 2019} 
\begin{itemize}
  \item 负责矿池的后端 API、后台、结算、清算统计、交易加速、消息推送模块等的设计和开发
  \item 主导开启和开发多个项目,包括重构矿池 Web 架构,结算拆分和风控,友商矿池合作,矿池后台改造,优化新币种开发,协助运维迁移至 kubernetes,跨区数据问题等
  \item 不断优化结算和财务报表模块,最终矿池的月度财务报表仅需两小时生成和校对完成
  \item 开发以及完善矿池监控,包括 stratum server 和 API,监控所有矿池的联合挖矿和块高、爆块地址、Job 下发效率等
\end{itemize}
\end{onehalfspacing}

\section{IT 技能}
\begin{onehalfspacing}
\begin{itemize}[parsep=0.5ex]
  \item 编程语言: Go,PHP,Python,C++
  \item 熟悉 Bitcoin,Ethereum,了解 Solidity,NFT,并在团队内部分享 ETH 2.0
  \item 熟悉 Flink,了解 ClickHouse,OLAP 数据分析
  \item 熟悉安全相关内容,熟悉常见安全漏洞,并在团队内部分享智能合约安全
  \item 数据库: MySQL,Redis
  \item 熟悉相关爬虫开发,了解一些反爬虫技术
\end{itemize}
\end{onehalfspacing}

\section{获奖以及实践经历}
\datedline{Flink Forward Asia 2020 实时数仓专场演讲,实时OLAP,从0 到1 }{ 2020 年}
\datedline{解魔方机器人, \textit{国家二等奖}, 全国第十一届博创杯比赛}{2015 年7 月}
\datedline{创办并运营CSDN高校俱乐部社团, 并获优秀社团}{2013 年 -- 2015 年}

\section{教育背景}
\datedsubsection{\textbf{山西农业大学} \textit{本科}\ 软件工程 }{2013 -- 2017}


\section{其他}
\begin{itemize}[parsep=0.5ex]
  \item 技术博客: \href{https://dbqf.xyz}{https://dbqf.xyz}
  \item GitHub: \href{https://github.com/dubuqingfeng}{https://github.com/dubuqingfeng}
  \item 微信: dubuqingfeng
\end{itemize}

\end{document}
