\documentclass{resume}
\usepackage{zh_CN-Adobefonts_external}
\usepackage{linespacing_fix} % disable extra space before next section
\usepackage{cite}

\begin{document}
\pagestyle{empty}

\name{高正炎}
\job{移动安全工程师}
\contactInfo{{moyishizhe@gmail.com}}{18404968744}{http://dbqf.xyz}
\section{项目经历}

\datedsubsection{\textbf{XSS-Receiver} ,
\href{https://github.com/dubuqingfeng/XSS-Receiver}{GitHub}}{2016 年5 月}
\role{设计、开发}{Python, JavaScript}
\begin{onehalfspacing}
\begin{itemize}
  \item 作为个人使用的 XSS 数据接收平台,可通过 Docker 快速部署
  \item 支持自定义 JS 等功能
\end{itemize}
\end{onehalfspacing}

\datedsubsection{\textbf{解魔方机器人} , \href{https://github.com/DigDream/RubiksCubeRobot}{GitHub}, \href{http://v.youku.com/v_show/id_XMTQ1NjExMTk3Mg==.html}{优酷}}{2015 年3 月 -- 2015 年5 月}
\role{设计、开发}{Android, arduino}
\begin{onehalfspacing}
\begin{itemize}
  \item 与网上大部分的相比,实现了一种更加稳定的结构。在平台上,后来用到了ARM,遇到了多路输出pwm的问题,经过耐心的查找资料,调试,后来在比赛中采用了定时器输出pwm的方式。
  \item Android客户端集成了计时器,打乱公式,复原纪录,魔坛资讯等实用功能
\end{itemize}
\end{onehalfspacing}

\datedsubsection{\textbf{大学圈圈客户端}, Android, \href{https://github.com/DigDream/shopnc-app}{GitHub}, \href{http://fir.im/vkeyandev}{Fir.im},
\href{http://www.wandoujia.com/apps/com.vkeyan.keyanzhushou}{豌豆荚}}{2015 年8 月 -- 2015 年10 月}
\begin{onehalfspacing}
\begin{itemize}
  \item {在开发过程中,遇到过gridview和listview或者两个listview在同一界面之间的高度问题并解决。}
\end{itemize}
\end{onehalfspacing}

\section{IT 技能}
\begin{onehalfspacing}
\begin{itemize}[parsep=0.5ex]
  \item 编程语言: Java > PHP > Python == C++
  \item 平台: Mac OS X, Linux
  \item 开发: Docker, Vagrant, Git
  \item 数据库: MySQL > Mongodb > Redis
  \item Web 安全: 熟悉常见的 Web 漏洞及其攻击技术
  \item 移动安全: 熟悉移动开发, 应用审计, 基础逆向
\end{itemize}
\end{onehalfspacing}

\section{获奖情况}
\datedline{解魔方机器人, \textit{国家二等奖}, 全国第十一届博创杯比赛}{2015 年7 月}
\datedline{2014联想茄子快传校园开发者大赛50强}{2014 年}

\section{实践经历}
\datedline{2015 年联想全国创业大赛 晋级华北大区决赛}{2015 年}
\datedline{社交订餐平台, 首届“晋商杯”大学生创业大赛二等奖}{2014 年}
\datedline{创办并运营CSDN高校俱乐部社团, 并获优秀社团}{2013 年 -- 2015 年}

\section{教育背景}
\datedsubsection{\textbf{山西农业大学}}{2013 -- 至今}
\textit{在读本科}\ 软件工程, 预计 2017 年 6 月毕业

\section{其他}
\begin{itemize}[parsep=0.5ex]
  \item 技术博客: \href{http://dbqf.xyz}{http://dbqf.xyz}
  \item GitHub: \href{https://github.com/dubuqingfeng}{https://github.com/dubuqingfeng}
  \item 微信: dubuqingfeng
\end{itemize}

\end{document}
